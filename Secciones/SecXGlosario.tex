\documentclass{report}
%% EN ESTA SECCI�N ES POSIBLE AGREGAR PAQUETES DE LIBRER�A QUE LE
%% AGREGUEN ALGUNA FUNCIONALIDAD EN PARTICULAR QUE DESEE AISLARSE EN
%% EL PRESENTE CAP�TULO
%%
%% EN EL CASO DE LOS AP�NDICES, EN OCASI�N ES PREFERIBLE UTILIZAR UN
%% ARCHIVOS INDEPENDIENTES POR CADA UNO DE ELLOS, SE DEBE EMPLEAR EL
%% COMANDO \include{} PARA PODER FACILITAR EL MANEJO DE ECUACIONES,
%% TABLAS Y FIGURAS EN DICHO ARCHIVO.
%%
\begin{document}

%{
%\addcontentsline{toc}{chapter}{Glosario}
%}

\newcommand{\mychapter}[2]{
    \setcounter{chapter}{#1}
    \setcounter{section}{0}
    \chapter*{#2}
    \addcontentsline{toc}{chapter}{#2}
}

\input{capitulosB}
\mychapter{7}{Glosario}
	
%\chapter*{Glosario}
%\addcontentsline{toc}{chapter}{Glosario}
%\appendix
%\renewcommand{\appendixname}{Glosario}
%\renewcommand{\appendixtocname}{Glosario}
%\renewcommand{\appendixpagename}{Glosario}
%\clearpage
%\addappheadtotoc
%\appendixpage



Transceptor: Dispositivo que adecua las se�ales el�ctricas a un nivel deseado.

Verilog: Lenguaje de programaci�n para dise�ar hardware.
interesen al lector, pero que ubicadas de este evitan ser una distracci�n durante la lectura.



\end{document}