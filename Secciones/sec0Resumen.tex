\documentclass{report}

\begin{document}


\addcontentsline{toc}{section}{Resumen}
\chapter*{}
\vspace*{-5.cm}
%\rhead{}
%\renewcommand{\headrulewidth}{0pt}
\begin{center}
\Large{\bfseries \parbox[c]{0.9\textwidth}{\centering \singlespacing{\titulo}}} \\
\vspace{10mm}
%\alumnos \\
\vspace{10mm}
\fontsize{14}{14}\selectfont \textbf{Resumen}
\end{center}

El protocolo CAN sirve para enviar y recibir informaci�n de forma eficiente entre los diversos nodos o componentes de un sistema automotriz. Es ampliamente usado en la industria automotriz y existen diversas versiones como CAN FD y CAN XL que permiten comunicaci�n Full duplex y un formato extendido respectivamente. La especificaci�n original ofrece funcionalidad a bajo nivel en las capas del modelo OSI, que son las capas f�sicas y de data link. Este protocolo ofrece varias oportunidades para ser mejorado y extendido en cualquiera de las capas superiores de ese modelo. Por ejemplo, protocolos de seguridad, estampas de tiempo, activaci�n por tiempo y para lograr un ancho de banda mas amplio. Todas estas nuevas caracter�sticas est�n implementadas en capas de superiores, es decir, de red, transporte, sesi�n y presentaci�n. Se presenta la implementaci�n de un nuevo modulo de propiedad intelectual que mejora el protocolo de comunicaci�n CAN. Esto ayuda a poder transmitir datos junto con otro protocolo como MQTT o poder transmitir informaci�n de manera que sea posible configurar el modulo a trav�s de alguna aplicaci�n o interfaz remota. Tambi�n se considera procesar informaci�n de diagn�sticos o aumentar la seguridad del protocolo a trav�s de encriptaci�n. Esta mejora tiene el mismo desempe�o que un modulo CAN normal sin modificar implicando que dicha mejora no supone una sobrecarga. Al mismo tiempo, se hace uso de la tecnolog�a open source para crear un circuito integrado de prop�sito especifico (ASIC) que implemente dicha funcionalidad. Esto se logra a trav�s del dise�o de hardware en lenguaje Verilog y generando los archivos necesarios llamados GDSII. Este archivo se usa para poder manufacturar el circuito integrado. Esto demuestra la efectividad de esta herramienta como alternativa a las herramientas de propietarias de gran costo. De esta manera se ofrece un dise�o f�sico al final del proyecto operando en una red CAN real.


\newpage


\end{document}
