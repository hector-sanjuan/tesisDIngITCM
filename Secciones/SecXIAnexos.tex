\documentclass{article}

%% EN ESTA SECCI�N ES POSIBLE AGREGAR PAQUETES DE LIBRER�A QUE LE
%% AGREGUEN ALGUNA FUNCIONALIDAD EN PARTICULAR QUE DESEE AISLARSE EN
%% EL PRESENTE CAP�TULO
%%
%% EN EL CASO DE LOS AP�NDICES, EN OCASI�N ES PREFERIBLE UTILIZAR UN
%% ARCHIVOS INDEPENDIENTES POR CADA UNO DE ELLOS, SE DEBE EMPLEAR EL
%% COMANDO \include{} PARA PODER FACILITAR EL MANEJO DE ECUACIONES,
%% TABLAS Y FIGURAS EN DICHO ARCHIVO.
%%

\begin{document}

\appendix
\renewcommand{\appendixname}{Anexo}
\renewcommand{\appendixtocname}{Anexos}
\renewcommand{\appendixpagename}{Anexos}
\clearpage
\addappheadtotoc
\appendixpage

\newcommand{\mychapter}[2]{
    \setcounter{chapter}{#1}
    \setcounter{section}{0}
    \chapter*{#2}
    \addcontentsline{toc}{chapter}{#2}
}

\input{capitulosB}
\mychapter{1}{Anexo A}

\begin{figure}[h]
	\centering
	\includegraphics[width=0.7\textwidth]{./pics/pic13.jpg}
	\caption{Modelos 3d de circuitos integrados.}
	\label{Modelo1} %Siempre incluir seguido de CAPTION.
\end{figure}

\begin{figure}[h]
	\centering
	\includegraphics[width=0.7\textwidth]{./pics/pic14.jpg}
	\caption{Otro modelo 3D.}
	\label{Modelo2} %Siempre incluir seguido de CAPTION.
\end{figure}

\begin{figure}[h]
	\centering
	\includegraphics[width=0.7\textwidth]{./pics/pic15.jpg}
	\caption{Placa Tinytapeout}
	\label{Tiny} %Siempre incluir seguido de CAPTION.
\end{figure}

\begin{figure}[h]
	\centering
	\includegraphics[width=0.7\textwidth]{./pics/pic16.jpg}
	\caption{Modelo OSI.}
	\label{OSI} %Siempre incluir seguido de CAPTION.
\end{figure}

\end{document}
