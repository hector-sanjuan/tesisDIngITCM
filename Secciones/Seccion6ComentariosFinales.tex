\documentclass[letterpaper,openright,12pt]{report}
%\usepackage[utf8]{inputenc}

%% EN ESTA SECCI�N ES POSIBLE AGREGAR PAQUETES DE LIBRER�A QUE LE
%% AGREGUEN ALGUNA FUNCIONALIDAD EN PARTICULAR QUE DESEE AISLARSE EN
%% EL PRESENTE CAP�TULO
%%
%% Ejemplo de un comando que sirve para abreviar el trabajo de
%% escritura permitiendo ser homog�neo al momento de escribir.
%\newcommand{\chla}{\emph{Chlamydomonas reinhardtii}}

\begin{document}
	
{
	\input{capitulos}
	\chapter{Comentarios finales}}

Sin duda, esta tesis ser� un proyecto interesante de emprender ya que en lo personal es algo nuevo para m�. En primera instancia, aprender� a usar una herramienta para crear circuitos integrados y ah� partir para implementar un protocolo de comunicaci�n. Aunque esto ya se ha hecho antes habr� muchas cosas que son diferentes como la tecnolog�a usada para crear el dise�o y la forma es que se verificara el mismo. Afortunadamente existen trabajos que han usado OpenLane de forma que ya est� probada la efectividad de la misma. En este sentido, ser� una buena experiencia aprender a dise�ar ASIC's y trabajar con el protocolo CAN al mismo tiempo y aprender a organizar un proyecto tan ambicioso.

\end{document}
